\documentclass[a4paper,12pt]{article}
\usepackage[utf8]{inputenc}
\usepackage[vietnamese]{babel}
\usepackage{graphicx}
\usepackage{array}
\usepackage{float}
\usepackage{enumitem}
\usepackage[T1]{fontenc} % Bắt buộc để hiển thị đúng chữ đ
\usepackage[utf8]{inputenc}
\usepackage{indentfirst}
\usepackage{listings}
\usepackage{xcolor}
\usepackage[table]{xcolor}
\usepackage{fontawesome5}
\usepackage{alltt}
\usepackage{longtable}
\usepackage[T1]{fontenc}
\usepackage{lmodern}
\usepackage{multirow}
\usepackage{booktabs}
\usepackage{placeins}
\usepackage{amsmath}
\usepackage{tabularx}
\usepackage{hyperref}
\usepackage{ragged2e}
\usepackage{fancyhdr}  % Thêm gói để tạo header, footer
\usepackage{titlesec}   % Để thay đổi kích thước của title section
\usepackage{tocloft}    % Để tùy chỉnh table of contents
\usepackage{geometry}
\geometry{a4paper, total={170mm,257mm}, left=20mm, top=20mm}

\pagestyle{fancy}
\fancyhf{}
\fancyhead[L]{\textbf{Báo Cáo IT4409}}
\fancyhead[R]{Nhóm 1}
\fancyfoot[L]{}
\fancyfoot[C]{\thepage}
\fancyfoot[R]{HUSTBuy}

\setlength{\headheight}{20pt}
\setlength{\footskip}{30pt}

% Định nghĩa màu sắc giống IDE Light Theme
\definecolor{commentgreen}{rgb}{0,0.4,0}
\definecolor{keywordblue}{rgb}{0,0,0.8}
\definecolor{stringred}{rgb}{0.6,0,0}
\definecolor{graynumber}{rgb}{0.5,0.5,0.5}

\lstset{
    backgroundcolor=\color{white},   
    commentstyle=\color{commentgreen},
    keywordstyle=\color{keywordblue}\bfseries,
    numberstyle=\tiny\color{graynumber},
    stringstyle=\color{stringred},
    basicstyle=\ttfamily\small,
    breakatwhitespace=false,         
    breaklines=true,                 
    captionpos=b,                    
    keepspaces=true,                 
    numbers=left,                    
    numbersep=5pt,                  
    showspaces=false,                
    showstringspaces=false,
    showtabs=false,                  
    tabsize=2,
    frame=single,
    rulecolor=\color{black!10},
    % Xử lý tiếng Việt bên trong code
    literate={á}{{\'a}}1 {à}{{\`a}}1 {ả}{{\?a}}1 {ã}{{\~a}}1 {ạ}{{.a}}1
             {é}{{\'e}}1 {è}{{\`e}}1 {ẻ}{{\?e}}1 {ẽ}{{\~e}}1 {ẹ}{{.e}}1
             {í}{{\'i}}1 {ì}{{\`i}}1 {ỉ}{{\?i}}1 {ĩ}{{\~i}}1 {ị}{{.i}}1
             {ó}{{\'o}}1 {ò}{{\`o}}1 {ỏ}{{\?o}}1 {õ}{{\~o}}1 {ọ}{{.o}}1
             {ú}{{\'u}}1 {ù}{{\`u}}1 {ủ}{{\?u}}1 {ũ}{{\~u}}1 {ụ}{{.u}}1
             {ý}{{\'y}}1 {ỳ}{{\`y}}1 {ỷ}{{\?y}}1 {ỹ}{{\~y}}1 {ỵ}{{.y}}1
             {đ}{{\textsubscript{d}}}1
}
% Đổi tên tiêu đề danh mục
\renewcommand{\lstlistlistingname}{Danh mục mã nguồn}

% Đổi tên nhãn tại mỗi đoạn code (Ví dụ: Mã nguồn 1: ...)
\renewcommand{\lstlistingname}{Mã nguồn}

\addto\captionsvietnamese{
  \renewcommand{\listfigurename}{Danh mục hình ảnh}
  \renewcommand{\listtablename}{Danh mục bảng biểu}
}
\begin{document}

\begin{titlepage}
\begin{center}
    \textbf{\Large ĐẠI HỌC BÁCH KHOA HÀ NỘI}\\[0.5em]
    \textbf{\large TRƯỜNG CÔNG NGHỆ THÔNG TIN \& TRUYỀN THÔNG}

    \includegraphics[width=0.2\textwidth]{divider.png}\\[2em]

    \includegraphics[width=0.7\textwidth]{soict.jpg}

    \vspace{1cm}

    \textbf{\Large GVHD: TS. Đỗ Bá Lâm}\\[1em]

    \textbf{\Large Chủ đề thực hiện:}\\[0.5em]
    {\Large \textbf{XÂY DỰNG WEB BÁN HÀNG ONLINE SỬ DỤNG KIẾN TRÚC MICROSERVICE}}\\[1em]

    \normalsize
    Mã lớp: \textbf{162307} \hspace{2em}
    Nhóm: \textbf{1} \hspace{2em} \\[1em]

    Môn học: \textbf{Công nghệ Web và dịch vụ trực tuyến} \hspace{0.5em} Mã môn học: \textbf{IT4409}\\[2em]

    \textbf{Các thành viên trong nhóm:}

    \vspace{1em}
    \setlength{\tabcolsep}{6pt} 
    \renewcommand{\arraystretch}{1.3}
    
    \begin{tabular}{|c|p{4cm}|c|p{7cm}|}
        \hline
        % Sử dụng \multicolumn để căn giữa tiêu đề
        \textbf{STT} & \multicolumn{1}{c|}{\textbf{Họ và tên}} & \textbf{MSSV} & \multicolumn{1}{c|}{\textbf{Nhiệm vụ}} \\
        \hline
        1 & Nguyễn Thế Anh & 20224921 & Trưởng nhóm, phân công và theo dõi tiến độ, phát triển Backend, xây dựng CI/CD \\ \hline
        2 & Hồ Lương An & 20224821 & Viết báo cáo, phát triển Frontend \\ \hline
        3 & Bùi Khắc Anh & 20225246 & Phát triển Backend, tester, phân tích yêu cầu \\ \hline
        4 & Lê Đình Hùng Anh & 20224917 & Làm slide, phát triển Frontend \\
        \hline
    \end{tabular}
    \end{center}

\vfill
\begin{center}
    \textit{Hà Nội, tháng 11 năm 2025}
\end{center}

\end{titlepage}

\renewcommand{\abstractname}{Lời nói đầu}
\begin{abstract}
Trong bối cảnh thương mại điện tử phát triển mạnh mẽ, việc xây dựng một hệ thống web bán hàng online hiệu quả, dễ bảo trì và mở rộng là vô cùng quan trọng. Báo cáo này trình bày quá trình thiết kế và triển khai website HUSTBuy - một nền tảng thương mại điện tử áp dụng kiến trúc Microservice.

Kiến trúc Microservice cho phép phân chia hệ thống thành các dịch vụ độc lập, giúp tăng tính linh hoạt, khả năng mở rộng và dễ dàng bảo trì. Qua đó, nhóm chúng em đã nghiên cứu và áp dụng các công nghệ hiện đại để xây dựng một giải pháp toàn diện từ phân tích, thiết kế đến triển khai.

Báo cáo bao gồm các nội dung chính: giới thiệu bài toán và mục tiêu hệ thống, phân tích và thiết kế chi tiết theo kiến trúc Microservice, các công nghệ được sử dụng, cùng với hình ảnh minh họa sản phẩm thực tế.

Nhóm xin chân thành cảm ơn thầy, TS. Đỗ Bá Lâm đã tận tình hướng dẫn trong suốt quá trình thực hiện project này.

\vspace{0.5cm}

\hspace*{10cm}  % Cách mép trái 10cm
\parbox{5cm}{
    \raggedleft
    \textbf{Nhóm 1}
}

\newpage
\tableofcontents
\listoffigures   % Hiển thị Danh mục hình ảnh
\listoftables    % Hiển thị Danh mục bảng biểu
\lstlistoflistings % Hiển thị Danh mục mã nguồn


\newpage
\section{Giới thiệu bài toán}

Trong thời đại công nghệ số hiện nay, thương mại điện tử đã trở thành xu hướng không thể thiếu trong cuộc sống. Người tiêu dùng ngày càng ưu tiên mua sắm trực tuyến nhờ sự tiện lợi, đa dạng sản phẩm và khả năng so sánh giá dễ dàng. Tuy nhiên, để xây dựng một hệ thống thương mại điện tử đáp ứng được nhu cầu ngày càng cao của người dùng về hiệu năng, tính sẵn sàng và khả năng mở rộng là một thách thức lớn.

Để giải quyết vấn đề này, kiến trúc Microservice đã nổi lên như một giải pháp tối ưu. Thay vì xây dựng hệ thống theo mô hình monolithic truyền thống, kiến trúc Microservice cho phép chia nhỏ ứng dụng thành các dịch vụ độc lập, mỗi dịch vụ đảm nhận một chức năng cụ thể. Điều này mang lại nhiều lợi ích như tăng khả năng mở rộng từng phần, dễ dàng bảo trì, triển khai linh hoạt và tối ưu hóa hiệu suất.

Dự án HUSTBuy được nhóm chúng em triển khai nhằm áp dụng kiến trúc Microservice vào thực tế, xây dựng một nền tảng thương mại điện tử hoàn chỉnh với đầy đủ các tính năng từ quản lý sản phẩm, giỏ hàng, thanh toán đến xác thực người dùng và quản lý đơn hàng.

\subsection{Bối cảnh và nhu cầu thực tiễn}

\textbf{Sự phát triển của thương mại điện tử:}

Theo báo cáo của các tổ chức nghiên cứu thị trường, ngành thương mại điện tử tại Việt Nam đang tăng trưởng với tốc độ hơn 20\% mỗi năm. Người tiêu dùng, đặc biệt là thế hệ trẻ, ngày càng quen thuộc với việc mua sắm trực tuyến. Điều này tạo ra nhu cầu cấp thiết về các nền tảng thương mại điện tử đáng tin cậy, nhanh chóng và dễ sử dụng.

\textbf{Thách thức của kiến trúc truyền thống:}

Các hệ thống thương mại điện tử truyền thống thường được xây dựng theo mô hình monolithic, nơi tất cả các chức năng được tích hợp trong một ứng dụng duy nhất. Mô hình này gặp phải nhiều hạn chế:

\begin{itemize}
    \item Khó mở rộng khi lưu lượng truy cập tăng cao
    \item Một lỗi nhỏ có thể làm sập toàn bộ hệ thống
    \item Khó khăn trong việc áp dụng công nghệ mới cho từng module
    \item Thời gian triển khai và bảo trì kéo dài
    \item Không tận dụng được tài nguyên một cách hiệu quả
\end{itemize}